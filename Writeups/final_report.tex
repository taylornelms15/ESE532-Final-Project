\documentclass{article}
\usepackage[margin=0.5in]{geometry}

\usepackage{listings}
\usepackage{enumitem}
\usepackage{appendix}
\usepackage{graphicx}


\title{ESE532 Project Final Report - Group}
\author{Ritika Gupta, Taylor Nelms, and Nishanth Shyamkumar}

\begin{document}

\maketitle


\section{Single ARM processor design}



\section{Ultra 96 design}
– Performance achieved and energy required
– Compression achieved
– Key design aspects: task decomposition, parallelism, mapping to Zynq resources, include diagrams to support
– Be clear where each component of the final design is performed (e.g., ARM, NEON vector, FPGA logic).
– Model to explain performance, area, and energy of design
– Current bottleneck preventing higher performance

All the components CDC, SHA, LZW and deduplication operate in hardware. 
Input data is read from network and it is processed in 2MB parts. 
// TODO: Detailed description about network reads

CDC performs chunking and the data is taken by SHA and LZW which operate parallely. The output of SHA which is 256 bit SHA value and the compressed output of LZW are both taken by deduplicate section to finally write the appropriate output to the output file. 
3 ARM cores are being used. One is used for reading the input data, one for processing the data and the third for writing the output to the output file.  


\section{10 Gbps design}

\section{Validation techniques}

\section{Key lessons learned}

\section{Design space exploration}

\section{Individual contribution}

\section{Academic code of integrity}
We, Ritika, Taylor and Nishanth, certify that I have complied with the University of Pennsylvania’s Code of Academic Integrity in completing this final exercise.


\end{document}
