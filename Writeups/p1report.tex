\documentclass{article}
\usepackage[margin=0.5in]{geometry}

\usepackage{listings}
\usepackage{enumitem}
\usepackage{appendix}
\usepackage{graphicx}


\title{ESE532 Project P1 Report}
\author{Ritika Gupta, Taylor Nelms, and Nishanth Shyamkumar}

\begin{document}

\maketitle


\begin{enumerate}
\item%1
Our group makeup is Ritika Gupta, Taylor Nelms, and Nishanth Shyamkumar.

\item%2
\begin{enumerate}[label=(\alph*)]
\item%a
We end up with $64ns$ to process each $64b$ word of input, which comes out to $76.8$ (so, $76$) cycles for a $1.2$GHz processor.

\item%b
By similar logic as the last question, with a $200$MHz clock, we end up with $12.8$ (so, $12$) cycles to process all of the input.

\end{enumerate}%2

\item%3
\begin{enumerate}[label=(\alph*)]
\item%a
\begin{enumerate}[label=(\roman*)]
\item%i
\textbf{Content-Defined Chunking}:
\begin{lstlisting}[language=python]

read input from secondary storage to a local buffer
hash arg is used to update window size chunk and store rabin fingerprint
  
  while input present:
    fread(buf, buf_len, 1, fp)
    len = buf_len
    while len >= 0:
      generatechunk(hash, buf, len)
      len -= hash.length
      sha256_hash(chunk)

generatechunk(hash, buf,len):
  for b in buf:
    rabin_slide(hash, b)  //Calculates rabin fingerprint on hash.window. 

    if(hash.length >= MIN_SIZE && (hash.digest & MASK) == 0 || hash.length >= MAX_SIZE) //MASK is a bitmask which detects 0 on LSB 20 bits.
      chunk is generated

      chunk.start = initial pointer
      chunk.length = hash.length
    
      reset_hash(hash)
      return

    hash.length++;  

\end{lstlisting}
\item%ii
\textbf{SHA-256}:
\begin{lstlisting}[language=python]

# RITIKA'S VERSION
hash[0:7] = initializeHashValues()
k[0:63] = initializeRoundConstants()

for each chunk
	for each 512_bit_subchunk m[]	# 64 byte sunchunk m[]
		for ( ; i < 64; ++i)	# For eevry byte in subchunk
			m[i] = SIG1(m[i - 2]) + m[i - 7] + SIG0(m[i - 15]) 
						+ m[i - 16];
		
		# call update function
		sha_update(data, len)

	# Compute hash for last incomplete chunk, if any
	if (last_subchunk < 448 bits)
		append 1
		append 0s to make 448 bits
	else
		append 1
		append 0s to make 512 bits
		sha_update(last_subchunk, 64)
		last_chunk[] = {0}
		
	append_0s_till_448_bit_subchunk()
	append_msg_len_in_last_64_bits()
	sha_update(last_subchunk)
	convert_hash_values_big_endian()		

sha_update(data, len)
	# initialize message schedule m[]
	for (i = 0, j = 0; i < 16; ++i, j += 4)
		m[i] = (data[j] << 24) | (data[j + 1] << 16) | (data[j + 2] << 8) | (data[j + 3]);
	for ( ; i < 64; ++i)
		m[i] = SIG1(m[i - 2]) + m[i - 7] + SIG0(m[i - 15]) + m[i - 16];

	# initialize working variables with previous hash values
	a = hash[0]
	b = hash[1]
	c = hash[2]
	d = hash[3]
	e = hash[4]
	f = hash[5]
	g = hash[6]
	h = hash[7]
		
	# update_working_variables()
	for ( ; i < 64; ++i)	# For every byte in subchunk
		t1 = h + EP1(e) + CH(e,f,g) + k[i] + m[i];
		t2 = EP0(a) + MAJ(a,b,c);
		h = g;
		g = f;
		f = e;
		e = d + t1;
		d = c;
		c = b;
		b = a;
		a = t1 + t2;
		
	# increment hash values by corresponding working variable
	hash[0] += a	
	hash[1] += b
	hash[2] += c
	hash[3] += d	
	hash[4] += e
	hash[5] += f
	hash[6] += g	
	hash[7] += h

digest = hash0 append hash1 append hash2 append hash3 append hash4 append hash5 append hash6 append hash7

\end{lstlisting}
Credit: Wikipedia
\item%iii
\textbf{Chunk Matching}:
\begin{lstlisting}[language=python]
if shaResult in chunkDictionary:
    send(shaResult)
else:
    send(LZW(rawChunk))
\end{lstlisting}
\item%iv
\textbf{LZW Encoding}:
\begin{lstlisting}[language=python]
table = {}
for i in range(256):
    table[i] = i
curPos = 256
STRING = Input.read()
while(True):
    CHAR = Input.read()
    if STRING + CHAR in table.values():
        STRING += CHAR
    else:
        Output.write(table[STRING])
        table[STRING + CHAR] = curPos
        curPos += 1
        STRING = CHAR
    if Input.isDone():
        break
\end{lstlisting}
Credit: https://www.dspguide.com/ch27/5.htm
\end{enumerate}%3a
\item%b % memory requirements
\textbf{Memory Requirements}
\begin{enumerate}[label=(\roman*)]
\item%i
\textbf{Content-Defined Chunking}:\newline
We'll need a rolling hash window's worth of working memory, spanning 16ish bytes.
Structure to store rabin fingerprint, chunk size that has been slid over.
2 2KB tables that is used by the Rabin algorithm.
\item%ii
\textbf{SHA-256}:\newline
Eight 32-bit span of memory to hold hash values, 
sixty four 32-bit span of memory for storing message schedule,
sixty four 32-bit span of memory for storing round constants,
ten 32-bit span of memory for working variables, and
64-byte SHA-block span of memory.

\item%iii
\textbf{Chunk Matching}:\newline
We'll want a table to store hash values for index purposes, which would require at least 8 bytes times the maximum number of chunks to be processed.

\item%iv
\textbf{LZW Encoding}:\newline
This is a somewhat tricky question given the associative memory involved, but it will be on the scale of roughly MAX\_CHUNK\_SIZE entries times 12 bits.

\end{enumerate}%3b

\item%c
\textbf{Computational Requirements}
\begin{enumerate}[label=(\roman*)]
\item%i
\textbf{Content-Defined Chunking}:\newline

\item%ii
\textbf{SHA-256}:\newline
Computation work per chunk: (Ignoring the index and loop iterator computations)
Prepare the message schedule m[i]: 16 * (3 ORs + 3 Shifts) = 96 operations
 						      (64 - 16) * (3 ADDS + 11 for SIG0 + 11 for SIG1) = 1200 operations
Update the working variables: 64 * (7 adds + 14 for EP0 + 14 for EP1 + 4 for CH + 5 for MAJ) = 2496 operations
Update hash values: 8 adds
Total no of operations = 96 + 1200 + 2496 + 8 = 3800 computations

\item%iii
\textbf{Chunk Matching}:\newline
The only real operation here is a dictionary lookup, so computation can sensibly be considered negligible.

\item%iv
\textbf{LZW Encoding}:\newline
With sensible dictionary lookup, there should be on the scale of one comparison operation per incoming byte of input, plus possibly a couple of additions as we loop through the incoming data.


\end{enumerate}%3c

\item%d
\textbf{Memory Access Requirements}
\begin{enumerate}[label=(\roman*)]
\item%i
\textbf{Content-Defined Chunking}:\newline

\item%ii
\textbf{SHA-256}:\newline
Memory operations per chunk(Ignoring local - BRAM reads and writes): 
Prepare the message schedule m[i]: 16 * (4 reads) = 64 
Initiate the working variables: 8 reads  
Update hash values : 8 writes
Total operations: 64 + 8 + 8 = 80 

\item%iii
\textbf{Chunk Matching}:\newline
One dictionary lookup should be required for each incoming hashed value; as such, we're looking at roughly 32 bytes of memory read (for reading the hash), and whatever memory costs are required after that for a dictionary lookup on that value.

\item%iv
\textbf{LZW Encoding}:\newline
With efficient encoding, there should be roughly one memory read and one memory write involved in devising the code for each incoming byte as part of LZW. This will, of course, be very dependent on the specific dictionary implementation.

\end{enumerate}%3d

\item%e

\end{enumerate}%3

\item%4
\begin{enumerate}[label=(\alph*)]
\item%a
The \textbf{LZW} and \textbf{SHA-256} operations can feasibly be done in parallel, as neither depends on the other. 
Once, SHA256 completes and tells whether the chunk already exists or not, the decision can be made whether to continue with LZW or discard its output.  
\item%b Task-Level
\textbf{Task-Level Parallelism}
\begin{enumerate}[label=(\roman*)]
\item%i
\textbf{Content-Defined Chunking}:\newline

\item%ii
\textbf{SHA-256}:\newline
Hash computation of each input chunk is independent of each other. But this is limited by the input coming from CDC, which will send input chunk by chunk.
For each input chunk, SHA256 works by dividing the input chunk into 512 bit subchunks and padding the last subchunk if it is less than 512. The computation of hash values for each subchunk 
is dependent on the previous hash computation. So, it inhibits parallelization of computation for each subchunk. Each subchunk has to go sequentially. But it does not need to wait for 
the entire input chunk to start computation. It can start as soon as it receives first 512 bits because padding only happens in the last subchunk, that's too only if it is less than 512 bits. 

\item%iii
\textbf{Chunk Matching}:\newline
There aren't many tasks here, so task-level parallelism seems a rather useless thing to pursue.

\item%iv
\textbf{LZW Encoding}:\newline
The overall task graph for LZW encoding is close enough to linear that there is not much reasonable task-level parallelism to be captured.

\end{enumerate}%4b Task-Level

\item%c Data-Level
\textbf{Data-Level Parallelism}
\begin{enumerate}[label=(\roman*)]
\item%i
\textbf{Content-Defined Chunking}:\newline
Any particular window of data could, feasibly, be rabin-fingerprinted at the same time; however, given the computational efficiency of doing a rolling hash sequentially, this seems ill-advised.
\item%ii
\textbf{SHA-256}:\newline
There is no data-level parallelism within the computation for each subchunk. There is data-level parallelism for each sub-chunk as different sub-chunks are to be given to each thread, but since 
one subchunk computation is dependent on previous subchunk computation, this parallelism can't be exploited.
\item%iii
\textbf{Chunk Matching}:\newline
With each table lookup result depending (potentially) on the last table entry, engaging in any parallelism here seems foolish.

\item%iv
\textbf{LZW Encoding}:\newline
With every incoming byte's code potentially dependent on the previous byte's code lookup results, there is no sensible data-level parallelism to be leveraged for LZW.

\end{enumerate}%4c Data-Level

\item%d Pipeline
\textbf{Pipeline Parallelism}
\begin{enumerate}[label=(\roman*)]
\item%i
\textbf{Content-Defined Chunking}:\newline
Any particular window of data could, feasibly, be rabin-fingerprinted at the same time; however, given the computational efficiency of doing a rolling hash sequentially, this seems ill-advised.
\item%ii
\textbf{SHA-256}:\newline
The entire SHA main loop operating on input chunks coming from CDC could be feasibly pipelined. II in this case would be equal to the no. of cycles it takes to complete hash computation for 1 chunk. The II is restrcited by dependencies of internal
variables within the subchunk computation. The depth of pipeline is no. of subchunks in an input chunk. 
\item%iii
\textbf{Chunk Matching}:\newline
There aren't really enough tasks here to pipeline sensibly.

\item%iv
\textbf{LZW Encoding}:\newline
If the end goal is to fill some kind of input/output buffers while dealing with streaming I/O data, any processes like that could be pipelined pretty well. For any of the internals, though, the logic should be atomic enough that the pipelining approach may not work optimally.

\end{enumerate}%4d Pipeline

\item%e

\end{enumerate}%4

\item%5
\begin{enumerate}[label=(\alph*)]
\item%a

\item%b

\item%c

\item%d

\item%e

\item%f

\end{enumerate}%5

\end{enumerate}%doc


\begin{appendices}
%\section{2m Filter.cpp}\label{2m}
%\lstinputlisting[language=C]{code/Filter.cpp}
%\section{1h mmult\_accel.cpp}\label{1hB}
%\lstinputlisting[language=C]{code/mmult_accel.cpp}
%\section{1h mmult\_accel.h}\label{1hC}
%\lstinputlisting[language=C]{code/mmult_accel.h}


\end{appendices}





\end{document}
