\documentclass{article}
\usepackage[margin=0.5in]{geometry}

\usepackage{listings}
\usepackage{enumitem}
\usepackage{appendix}
\usepackage{graphicx}


\title{ESE532 Project P2 Report}
\author{Ritika Gupta, Taylor Nelms, and Nishanth Shyamkumar}

\begin{document}

\maketitle


\section{Design Space Axes}
\begin{enumerate}
\item%1

\textbf{Axis:} $S$, Number of SHA-256 hardware units
\newline
\textbf{Challenge:} Improving throughput of hashing step
\newline
\textbf{Opportunity:} Send chunks to rotating SHA unit index to allow for parallel execution
\newline
\textbf{Continuum:} Anywhere from $1$ to however many of our hardware SHA units will fit on the FPGA
\newline
\textbf{Equation for Benefit:} \texttt{Throughput}$\left(S\right)=S*\texttt{singleSHAUnitThroughput}$

\item%2

\textbf{Axis:} $L$, Number of LZW hardware units
\newline
\textbf{Challenge:} Improving throughput of LZW step
\newline
\textbf{Opportunity:} Send chunks to rotating LZE unit index to allow for parallel execution
\newline
\textbf{Continuum:} Anywhere from $1$ to however many of our hardware LZW units will fit on the FPGA (BRAM likely limiting factor)
\newline
\textbf{Equation for Benefit:} \texttt{Throughput}$\left(L\right)=S*\texttt{singleLZWUnitThroughput}$

\item%3

\textbf{Axis:} $Z$, Design choice for LZW hash table unit
\newline
\textbf{Challenge:} Allow for efficient access of code-table for LZW step while fitting within hardware specifications
\newline
\textbf{Opportunity:} Use trees or associative memories (or both) to allow for low cycle count for finding relevant table entry
\newline
\textbf{Continuum:} $Z\in\{$Tree with Dense RAM, Tree with Fully Associative Memory, Tree with Tree, Tree with Hybrid$\}$
\newline
\textbf{Equation for Benefit:} Slide 65 from Day 17 has the relevant tradeoff chart, with implied \texttt{implementation\_complexity} parameter to consider.

\item%4

\textbf{Axis:} $H$, Number of bits in hash of SHA value for storing SHA values
\newline
\textbf{Challenge:} Effectively storing mapping between SHA values of previous chunks and the chunk index
\newline
\textbf{Opportunity:} Tune hash table size to reduce conflicts but also remain compact
\newline
\textbf{Continuum:} Could be any small number of bits (call it $5$ as a low value) through $256$ for the full SHA value.
\newline
\textbf{Equation for Benefit:}
\[
\texttt{numRows }C=2^H
\]
\[
\texttt{probCollision}={N \choose m}\left(\frac{1}{C}\right)^m \left(1 - \frac{1}{C}\right)^{N-m}
\]

\item%5


\end{enumerate}


\section{Teamwork}


\begin{appendices}
%\section{2m Filter.cpp}\label{2m}
%\lstinputlisting[language=C]{code/Filter.cpp}
%\section{1h mmult\_accel.cpp}\label{1hB}
%\lstinputlisting[language=C]{code/mmult_accel.cpp}
%\section{1h mmult\_accel.h}\label{1hC}
%\lstinputlisting[language=C]{code/mmult_accel.h}


\end{appendices}





\end{document}
