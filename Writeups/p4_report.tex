\documentclass{article}
\usepackage[margin=0.5in]{geometry}

\usepackage{listings}
\usepackage{enumitem}
\usepackage{appendix}
\usepackage{graphicx}


\title{ESE532 Project P4 Report - Group}
\author{Ritika Gupta, Taylor Nelms, and Nishanth Shyamkumar}

\begin{document}

\maketitle


\section{Deduplication and compression}
\begin{enumerate}
\item%a
\textbf{Throughput}: The throughput is 10Mbps. The throughput decreased from 15 to 10 by putting rabin, SHA also in hardware because 
\newline

\item%b
\textbf{Compression status}: Current comporession status for vmlinuz.tar is 44% for an orginal file size of 66MB and it compressed it to 37MB. For little prince, the compression ratio is 67% with the original file size of 14KB and the compressed file size 4.5KB. For a file not stuitable for compression like franklin, the compression ratio is 21%  for an original file size of 390K and the compressed file size 308K.
\newline

\item%c
\textbf{Validations}: We use decoder to produce the uncompressed file and then compare it with the original uncompressed file and they match.
\newline

\item%d
\textbf{Design space}
\newline

\item%e
\textbf{Techniques for speedup}: We put all three components in harwdare this time to increase the speed. It saves time to send data from PS to PL and then back. 
\newline


\item%f
\textbf{Perfomance model}
\newline


\item%b
\textbf{Task distribution}: 
\newline

\section{Code}

Included in different turn-in location.

\section{Binaries}

Included in different turn-in location.


\begin{appendices}
%\section{2m Filter.cpp}\label{2m}
%\lstinputlisting[language=C]{code/Filter.cpp}
%\section{1h mmult\_accel.cpp}\label{1hB}
%\lstinputlisting[language=C]{code/mmult_accel.cpp}
%\section{1h mmult\_accel.h}\label{1hC}
%\lstinputlisting[language=C]{code/mmult_accel.h}


\end{appendices}





\end{document}
